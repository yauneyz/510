\documentclass{article}
\title{Homework 11}
\author{Zac Yauney}

\usepackage{amssymb}
\usepackage{amsmath}
\usepackage{mathrsfs}

% Custom commands and operators
\newcommand\range{\mathscr{R}}
\newcommand\n{\mathcal{N}}
\newcommand\F{\mathbb{F}}
\newcommand\C{\mathbb{C}}
\newcommand\Z{\mathbb{Z}}
\newcommand\E{\mathbb{E}}
\newcommand\Q{\mathbb{Q}}
\newcommand\R{\mathbb{R}}
\newcommand\bo{\mathcal{O}}
\newcommand\ip[2]{\langle #1,#2\rangle}
\newcommand\norm[1]{||#1||}
\DeclareMathOperator{\tr}{tr}
\DeclareMathOperator{\rank}{rank}
\DeclareMathOperator{\proj}{proj}
\DeclareMathOperator{\Skew}{Skew}
\DeclareMathOperator{\Sym}{Sym}
\DeclareMathOperator{\Res}{Res}
\DeclareMathOperator{\spn}{span}
\DeclareMathOperator{\Var}{Var}
\DeclareMathOperator{\argmin}{argmin}
\DeclareMathOperator{\argmax}{argmax}


\begin{document}
\maketitle
\paragraph{1}
The text shows that $A^\dagger$ has the SVD $A^\dagger=V\Sigma_{A}^{-1} U^*$. Compare this to $A_1^{-1}=V \Sigma_{A_1}^{-1}U^*$. The unitary matrices all have 2-norm of 1, and the 2-norm is equivalent to the largest singular value, so we have
\[
	||A^\dagger||_2 = \frac{1}{\sigma_\text{min}(A)}
\]
\[
	||A_1^{-1}||_2 = \frac{1}{\sigma_\text{min}(A_1)}
\]
Then we only need show that $\sigma_\text{min}(A) \ge \sigma_\text{min}(A_1)$. This follows from the fact that the singular values for $A$ are the square roots of
\[
A^*A = \begin{bmatrix}
	A_1 & A_2
\end{bmatrix}
\begin{bmatrix}
A_1 \\
A_2
\end{bmatrix}
= A_1^*A_1+A_2^*A_2 \ge A_1^*A_1
\]
which is greater than the singular values for $A_1$, the square roots of the eigenvalues of
\[
A_1^*A_1
\].

This shows that $A$ has larger singular values, and thus $A^\dagger$ has a smaller 2-norm than $A_1^{-1}$.



\end{document}
